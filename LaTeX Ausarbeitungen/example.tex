\documentclass[ngerman,landscape,twocolumn]{adtexsheet}

 %for pseudocode
 %http://tug.ctan.org/macros/latex/contrib/algorithm2e/doc/algorithm2e.pdf
\usepackage[ruled,noend,noline,nofillcomment,linesnumbered,]{algorithm2e}
\DontPrintSemicolon
\SetKwFor{For}{for }{}{}
\SetKwFor{While}{while }{}{}
\SetKwIF{If}{ElseIf}{Else}{if}{}{else if}{else}{}
\SetKw{Return}{return}
\newcommand{\To}{\KwTo}
\newcommand{\swap}{\leftrightarrow}
\newcommand*{\CommentInLine}{\tcc*[f]}

\exnumber{1}
%Teilnehmer
%<Name> & <Gruppennummer> & <Kreuz Aufgabe 1> & <Kreuz Aufgabe 2> & <Kreuz Aufgabe 3> & <Kreuz Aufgabe 4>
\participantOne{Anton&1&x&x&-&-}
\participantTwo{Berta&1&x&x&x&-}
\participantThree{Charlie&1&-&x&-&-}
\participantFour{-&-&-&-&-&-}

% sheet specific notions and notation,
%drawing automatons
\usepackage{tikz}
\tikzstyle{state}=[draw,circle,fill=gray!35, inner sep = 0pt, 
minimum width = 22pt]
\tikzstyle{accept}=[state,double]

\begin{document}

	\begin{question}
		\begin{enumerate}
			\item Beispiel-Algorithmus:
				\begin{algorithm}
					\tcc{For-Schleife}
					\For{$i \gets 1$ \To n}{
						$A[1]\swap A[2]$ \tcc*{Vertauschung}
					}
					\tcc{While-Schleife}
					\While{$i < n$}{
						\If(\tcc*[f]{Kaffee ist besser}){coffee $>$ tea}{drink coffee}
						\ElseIf(\tcc*[f]{Tee ist besser}){tea $>$ coffee}{drink tea}
						\Else(\tcc*[f]{gleich gut}){drink both}
					}
					\KwSty{superkalifragilistisch} \tcc*{custom keyword}
					\Return 42
					\caption{Beispiel(n,A,tea,coffee)}
			\end{algorithm}
		
			\item \dots
		\end{enumerate}
	\end{question}

	\newpage

	\begin{question}
		\dots
	\end{question}

\end{document}